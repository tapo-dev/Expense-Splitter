\documentclass[12pt,a4paper]{article}
\usepackage[utf8]{inputenc}
\usepackage[czech]{babel}
\usepackage[T1]{fontenc}
\usepackage{lmodern}
\usepackage{geometry}
\usepackage{graphicx}
\usepackage{enumitem}
\usepackage{listings}
\usepackage{xcolor}
\usepackage{fancyhdr}
\usepackage{hyperref}
\usepackage{float}

% Nastavení stránky
\geometry{margin=2.5cm}
\pagestyle{fancy}
\fancyhf{}
\fancyhead[L]{RoommateApp}
\fancyfoot[C]{\thepage}

% Nastavení pro kód
\lstset{
    basicstyle=\ttfamily\footnotesize,
    breaklines=true,
    frame=single,
    backgroundcolor=\color{gray!10}
}

% Hyperlinks
\hypersetup{
    colorlinks=true,
    linkcolor=blue,
    urlcolor=blue,
    citecolor=blue
}

\title{\textbf{RoommateApp} \\ 
       \large Aplikace pro správu výdajů spolubydlících}
\author{Tadeáš Pohlídal}
\date{\today}

\begin{document}

\maketitle
\newpage

\tableofcontents
\newpage

\section{Přehled aplikace}

\textbf{RoommateApp} je mobilní aplikace vytvořená v .NET MAUI pro správu společných výdajů mezi spolubydlícími. Umožňuje uživatelům vytvářet skupiny, přidávat výdaje, automaticky počítat dluhy a spravovat jejich splacení s notifikačním systémem.

\subsection{Klíčové funkcionality}
\begin{itemize}
    \item Správa uživatelských účtů s bezpečnou autentizací
    \item Vytváření a správa skupin spolubydlících
    \item Přidávání výdajů
    \item Automatický výpočet dluhů mezi členy
    \item Systém notifikací o splacených dluzích
    \item Přehledné zobrazení výdajů a dluhů
\end{itemize}

\section{Architektura aplikace}

\subsection{Technologie}
\begin{itemize}
    \item \textbf{.NET 8}
    \item \textbf{.NET MAUI}
    \item \textbf{Entity Framework Core}
    \item \textbf{SQLite}
\end{itemize}

\subsection{Vrstvová architektura}

Aplikace je navržena ve čtyřvrstvové architektuře:

\begin{enumerate}
    \item \textbf{Presentation Layer (Prezentační vrstva)}
    \begin{itemize}
        \item Views -- XAML stránky aplikace
        \item ViewModels -- business logika pro UI
    \end{itemize}
    
    \item \textbf{Application Layer (Aplikační vrstva)}
    \begin{itemize}
        \item Services -- aplikační služby (AuthService, NotificationService)
        \item Factories -- továrny pro vytváření objektů (Factory pattern)
        \item Auth -- logika pro autentizaci a autorizaci
    \end{itemize}
    
    \item \textbf{Domain Layer (Doménová vrstva)}
    \begin{itemize}
        \item Models -- doménové entity (Uzivatel, Skupina, Vydaj, Dluh)
        \item Strategies -- Strategy pattern
        \item Observers -- Observer pattern
    \end{itemize}
    
    \item \textbf{Infrastructure Layer (Infrastrukturní vrstva)}
    \begin{itemize}
        \item Data -- AppDbContext a databázové konfigurace
        \item DbContext -- Entity Framework context
    \end{itemize}
\end{enumerate}

\section{UML diagramy}

\subsection{Use Case diagram}
\begin{figure}[H]
    \centering
    \includegraphics[width=0.9\linewidth]{Use_case.png}
    \label{fig:use-case}
\end{figure}

\subsection{Class diagram}
\begin{figure}[H]
    \centering
    \includegraphics[width=1\linewidth]{Class.png}
    \label{fig:class}
\end{figure}

\subsection{Sequence diagram}
\begin{figure}[H]
    \centering
    \includegraphics[width=1\linewidth]{Sequential.png}
    \label{fig:sequential}
\end{figure}

\subsection{State Machine diagram}
\begin{figure}[H]
    \centering
    \includegraphics[width=1\linewidth]{State_machine.png}
    \label{fig:state_machine}
\end{figure}

\subsection{Activity diagram}
\begin{figure}[H]
    \centering
    \includegraphics[width=0.75\linewidth]{Activity.png}
    \label{fig:activity}
\end{figure}

\section{Návrhové vzory}

\subsection{Strategy Pattern}
\textbf{Účel}: Flexibilní výpočet dluhů podle různých strategií
\begin{itemize}
    \item \texttt{IVypocetDluhuStrategy} -- rozhraní strategie
    \item \texttt{RovnomerneRozdeleniStrategy} -- rovnoměrné dělení částky
    \item \texttt{VahoveRozdeleniStrategy} -- dělení podle vah členů
\end{itemize}

\textbf{Použití}: \texttt{SpravceUctu} může za běhu měnit strategii výpočtu

\subsection{Observer Pattern}
\textbf{Účel}: Notifikace o změnách stavu dluhů
\begin{itemize}
    \item \texttt{ISubject} / \texttt{IObserver} -- základní rozhraní
    \item \texttt{Dluh} implementuje \texttt{ISubject}
    \item Konkrétní observeři: \texttt{EmailNotifier}, \texttt{SmsNotifier}, \texttt{InAppNotifier}, \texttt{ConsoleNotifier}
\end{itemize}

\textbf{Použití}: Při označení dluhu jako splacený se automaticky rozešlou notifikace

\subsection{Factory Pattern}
\textbf{Účel}: Vytváření instancí notifikačních služeb
\begin{itemize}
    \item \texttt{INotifierFactory} -- rozhraní factory
    \item \texttt{NotifierFactory} -- konkrétní implementace
    \item Dynamické vytváření notifierů podle typu
\end{itemize}

\textbf{Použití}: \texttt{NotificationService} používá factory pro vytváření aktivních notifierů

\subsection{MVVM Pattern}
\textbf{Účel}: Oddělení prezentační a business logiky (Jenom částečně implementované)
\begin{itemize}
    \item \texttt{BaseViewModel} -- základní třída pro view modely
    \item \texttt{RelayCommand} / \texttt{AsyncRelayCommand} -- implementace ICommand
    \item Data binding mezi Views a ViewModels
\end{itemize}

\section{Databázový model}

\subsection{Hlavní entity}
\begin{itemize}
    \item \textbf{Uzivatel} -- uživatelské účty s hashem hesel
    \item \textbf{Skupina} -- skupiny spolubydlících
    \item \textbf{Clenstvi} -- M:N vazba mezi uživateli a skupinami
    \item \textbf{Vydaj} -- společné výdaje ve skupinách
    \item \textbf{Dluh} -- automaticky generované dluhy z výdajů
\end{itemize}

\subsection{Klíčové vztahy}
\begin{itemize}
    \item Uživatel <-> Skupina (M:N přes Clenstvi)
    \item Skupina -> Vydaj (1:N)
    \item Vydaj -> Dluh (1:N)
    \item Uživatel -> Dluh (1:N jako dlužník i věřitel)
\end{itemize}

\section{Bezpečnost}

\subsection{Přihlašování}
\begin{itemize}
    \item Hesla jsou šifrovaná pomocí BCryptu
    \item Aplikace si pamatuje přihlášeného uživatele
    \item Při prvním spuštění se automaticky vytvoří testovací účty (V Homepage je zakomentované tlačítko pro reset databáze pro testovací purposes)
\end{itemize}

\subsection{Oprávnění uživatelů}
\begin{itemize}
    \item Každá skupina má svého administrátora
    \item Pouze administrátor může mazat skupinu nebo upravovat členy
    \item Dluh může označit jako splacený pouze věřitel
    \item Všechna zadávaná data se kontrolují před uložením
\end{itemize}

\section{Testování a kvalita}

\subsection{Implementované kontroly}
\begin{itemize}
    \item \textbf{Validace dat} na všech vrstvách
    \item \textbf{Exception handling} s uživatelsky přívětivými hláškami
    \item \textbf{Asynchronní operace} pro databázové volání
    \item \textbf{Memory management} s properly disposed objekty
\end{itemize}

\subsection{Potenciální rozšíření}
\begin{itemize}
    \item Dodělat přidávání přátel
    \item Unit testy pro business logiku
    \item Refactoring
    \item Integration testy pro databázové operace
    \item UI testy pro kritické user journeys
\end{itemize}

\section{Závěr}

Aplikace RoommateApp je funkční mobilní aplikace pro správu výdajů spolubydlících (a teoreticky jakýchkoliv výdajů). Úspěšně implementuje všechny požadované návrhové vzory a splňuje zadání projektu. Pro další rozšíření by bylo vhodné dokončit MVVM implementaci a přidat testování.

\end{document}